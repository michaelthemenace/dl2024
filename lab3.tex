\documentclass{article}

\title{Labwork 3: Logistic Regression}

\begin{document}

\maketitle

\setlength\parindent{0pt}

\section{Introduction}

Logistic Regression:
Logistic regression is a statistical method used for binary classification problems. It models the 
probability that a given input belongs to a particular class using a logistic function. Unlike linear 
regression, logistic regression predicts a probability value between 0 and 1, which can then be 
thresholded to classify the input.\\
\\
In this lab, we implemented a logistic regression algorithm using gradient descent to minimize the 
loss function. The loss function used is the Binary Cross-Entropy (BCE), which measures the difference 
between the predicted probabilities and the actual class labels.

\section{Implementation}

The implementation of logistic regression in this lab uses the gradient descent optimization algorithm 
to iteratively update the weights $w_0$, $w_1$, and $w_2$ to minimize the loss function.\\
\\
The logistic regression model predicts the probability of the positive class for a given input $(x_1, x_2)$ 
as follows:\\
\[\hat{y} = \sigma(w_1 x_1 + w_2 x_2 + w_0)\]
where $\sigma(z)$ is the sigmoid function defined as:
\[\sigma(z) = \frac{1}{1 + e^{-z}}\]

The Binary Cross-Entropy loss function for a single data point $(x_1, x_2, y)$ is defined as:\\
\[L = -\left[y \log(\hat{y}) + (1 - y) \log(1 - \hat{y})\right]\]

The weights are updated iteratively using the learning rate $r$:\\
\[w_0 \leftarrow w_0 - r \frac{\partial L}{\partial w_0}\]
\[w_1 \leftarrow w_1 - r \frac{\partial L}{\partial w_1}\]
\[w_2 \leftarrow w_2 - r \frac{\partial L}{\partial w_2}\]

The algorithm stops after a fixed number of iterations or when the loss converges below a specified threshold.

\section{Evaluation}

The program was tested with a dataset loaded from a CSV file. The user provides the learning rate $r$ 
and initial values for $w_0$, $w_1$, and $w_2$. The program outputs the updated weights, predicted 
probabilities, and the loss for each iteration.\\
\\
For example, testing with a dataset containing loan approval data, the program iteratively adjusts the 
weights to minimize the loss. Below is an example of the output:

\begin{verbatim}
1th iteration:
w0: -0.25, w1: 3.75, w2: 5.0, yi_pred: 0.0 loss: -0.6931471805599453
...
Final w0: -13.749999999999888, 
Final w1: 27.224999999999717, 
Final w2: 135.99999999999955, 
Final loss: 250.52180804312684
\end{verbatim}

\section{Conclusion}

In this labwork, we implemented a logistic regression algorithm using gradient descent. The algorithm 
successfully minimized the Binary Cross-Entropy loss function and found the optimal weights for the 
given dataset.\\
\\
An important observation is that the choice of learning rate $r$ significantly affects the convergence 
speed and stability of the algorithm. Proper tuning of $r$ is essential to achieve optimal results 
without overshooting or slow convergence.

\end{document}