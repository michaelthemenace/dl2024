\documentclass{article}


\title{Labwork 1: Gradient Descent}

\begin{document}

\maketitle

\setlength\parindent{0pt}

\section{Introduction}

Gradient descent: definition

Gradient descent is an optimization algorithm used to minimize a function by iteratively moving in the direction of the steepest descent, as defined by the negative of the gradient.

\section{Implementation}

The implementation of gradient descent in this lab minimizes the function $f(x)$. 
The algorithm starts with an initial value of $x$, a learning rate $r$, and a convergence threshold $\epsilon$. 
It iteratively updates $x$ using the formula $x = x - r \cdot f'(x)$, where $f'(x)$ is the gradient of $f(x)$. 
The process continues until $f(x)$ becomes less than or equal to $\epsilon$, at which point the algorithm returns the final values of $x$ and $f(x)$.

\section{Evaluation}

Test with $y = x^4$

Output of my program:\\
\\
Enter initial x: 2\\
Enter learning rate r: 0.05\\
Enter convergence threshold epsilon: 0.005\\
Final x: 0.3999999999999999, Final f(x): 0.025599999999999977\\
\\
The x and f(x) steps:
Current x: 0.3999999999999999, f(x): 0.025599999999999977\\
Current x: 0.38719999999999993, f(x): 0.022477157800345586\\
Current x: 0.3755898978303999, f(x): 0.019900116114290756\\
Current x: 0.3649931718645291, f(x): 0.017747572531879738\\
Current x: 0.35526829266132753, f(x): 0.015930367528179314\\
Current x: 0.3463002153759361, f(x): 0.014381727201132847\\
Current x: 0.3379942851093202, f(x): 0.013050808848442437\\
Current x: 0.3302717824378793, f(x): 0.011898326472229414\\
Current x: 0.3230666095439659, f(x): 0.010893521523512159\\
Current x: 0.3163227856997814, f(x): 0.010012024563155282\\
Current x: 0.30999252748521344, f(x): 0.009234319577443924\\
Current x: 0.3040347583400302, f(x): 0.008544623793691794\\
Current x: 0.2984139379836053, f(x): 0.007930058940763027\\
Current x: 0.2930991333237783, f(x): 0.00738003016633253\\
Current x: 0.28806327389789593, f(x): 0.0068857550438280605\\
Current x: 0.2832825498919133, f(x): 0.006439902494252792\\
Current x: 0.27873592144853765, f(x): 0.006036313180470917\\
Current x: 0.27440471566128116, f(x): 0.005669780969569371\\
Current x: 0.27027229326037167, f(x): 0.005335880645204224\\
Current x: 0.2663237711378589, f(x): 0.005030830985947597\\
Current x: 0.2625457899501838, f(x): 0.0047513851304683925\\
Final x: 0.2625457899501838, Final f(x): 0.0047513851304683925\\
\\
How ever, for the learning rate of 0.2, the program gives:\\
OverflowError: (34, 'Numerical result out of range')\\
\\
The reason for this is that the learning rate is too high, causing the algorithm to overshoot the minimum and produce an overflow error.

\section{Conclusion}

In this labwork, I implemented the gradient descent algorithm to minimize the function $f(x) = x^4$.
The result is that the algorithm converged to a minimum value of $f(x)$ at $x \approx 0.4$ with a final function value of approximately $0.0256$.\\
\\
An interesting finding is that the choice of learning rate significantly affects the convergence speed and stability of the algorithm. A smaller learning rate leads to slower convergence, while a larger one may cause divergence.

\end{document}
